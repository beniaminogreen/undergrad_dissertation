\author{Beniamino Green}
\title{Pre-Registration for Undergrad Dissertation}

\documentclass{article}

\usepackage{fancyhdr}
\usepackage{tcolorbox}
\usepackage{mathtools}
\usepackage{float}
\usepackage{threeparttable}
\usepackage{tabularx}
\usepackage{array}
\usepackage{multirow}
\usepackage{listings}
\usepackage{graphicx}
\usepackage{setspace}
\usepackage{verbatim}
\usepackage{amssymb}
\usepackage[backend=biber, style=authoryear-icomp,doi=false,isbn=false,url=false]{biblatex}
\addbibresource{$BIB}

\addtolength{\oddsidemargin}{-.5in}
\addtolength{\evensidemargin}{-.5in}
\addtolength{\textwidth}{1in}
\addtolength{\topmargin}{-.5in}
\addtolength{\textheight}{1in}

\pagestyle{fancy}
\fancyhf{}
\rhead{Beniamino Green}
\lhead{Pre-Registration for Undergrad Dissertation}
\rfoot{\thepage}
\begin{document}
\maketitle{}
\textit{This paper uses Google search terms containing offensive language as a proxy measure for racial animus. Following the practice of similar works \parencites{Stephens_Davidowitz_2014}{Chae_2015}{Chae_2018}{Isoya_2021}, I use coded language to refer to these terms. The words themselves can be found in Table \ref{coded}}

\section{Introduction}

Previous research has demonstrated that Sinclair acquisition of a network is associated with a sharp rightwards shift in its coverage \parencite[][]{Martin_2019}.
However, a quantitative analysis of the effect of Sinclair ownership on coverage of racial issues has not been conducted.

There is significant evidence to suggest that Sinclair ownership does push stations to run more racially conservative stories than they otherwise would.
This year, the Sinclair corporation drew ire for a series of ``must-run'' segments on police violence following the murder of George Floyd pushing the ``black-on-black violence'' canard and advocating for a military response to the protests.

Sinclair additionally has a history of hiring disgraced Fox News hosts

\section{Methodology}

Following \cite{Stephens_Davidowitz_2014}, I use Google trends data as a proxy measure for racial animus.

Google trends data provide a sample of the searches made by Google users over a certain time frame in a region.
The popularity of a search in a region at a certain time is expressed as a search score, calculated as the following:

\[
\text{Search Score}_{ijk} = [\frac{\text{number of searches for } i \text{ during time } j \text{ in region } k }{\text{maximum number of searches for } i \text{ over any month } \text{ in region } k }] \times 100
\]


This poses a problem for a difference-in-difference approach, as the maximum number of searches for a word in any time period changes between areas.
As the maximum monthly searches for a term differs between regions, A score of 50 in one region might correspond to 500 searches, while the same score might correspond to 5000 searches in another area.

Similarly

\begin{table}[!htb]
    \caption{Example Raw Data From Google Trends}
    \begin{minipage}{.6\linewidth}
        \centering
        \begin{tabular}{|c|c|c|c|c|c|}
            \hline
            DMA Code           & shoe & fruit & horse & cup & smart \\ \hline
            803   & 23          & 100     & 5           & 10            & 27         \\ \hline
            616   & 26          & 86      & 3           & 15            & 45         \\ \hline
            617         & 32          & 94      & 6           & 24            & 31 \\ \hline
        \end{tabular}
        \caption{Search trends data for words 1-5}
    \end{minipage}%
    \begin{minipage}{.6\linewidth}
        \centering
        \begin{tabular}{|c|c|c|c|c|c|}
            \hline
            DMA Code           & smart & pen & waltz & gnome & boots \\ \hline
            803   & 14         & 32    & 5       & 3       & 100     \\ \hline
            616   & 22         & 54    & 13      & 2       & 88      \\ \hline
            617         & 16         & 50    & 12      & 1       & 96      \\ \hline
        \end{tabular}
        \caption{Search trends data for words 5-9}
    \end{minipage}
\end{table}



\printbibliography{}
\appendix
\section{Codings for Offensive Words}
\begin{centering}
\begin{table}[]
    \begin{tabular}{|l|l|} \\ \hline
    Code & Word \\ \hline
         &      \\ \hline
         &      \\ \hline
\end{tabular}
\end{table}
\end{centering}


\end{document}

